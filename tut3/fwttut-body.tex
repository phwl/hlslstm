
\section{Introduction}
The discrete Walsh transform (DWT) takes $O(N^2)$ addition/subtraction
operations to execute. This can be reduced to $O(N \log_2 N)$ using
the fast Walsh transform (FWT). The algorithm, explained in
reference~\cite{Shanks:69} is analogous to the Cooley-Tukey algorithm
for the fast Fourier transform.

\section{Laboratory Questions}
\begin{enumerate}
\item Fast Walsh transform processor (30\%). 
Make a combinatorial, parallel implementation of an $N=64$ FWT processor
for the Altera Cyclone V 5CSEMA5 FPGA used in the DE1-SoC board.
Your inputs should be 16-bit integers in two's complement form, and
your output represented as a two's complement fraction with sufficiently
large wordlength that overflow cannot occur. 

Create a set of random test vectors and verify that your design is
correct via simulation. The FPGA design tools report the maximum
clock rate, $f_{max}$ which can be achieved by your design. What is this
value? The 
maximum throughput is thus $2Nf_{max}$ bytes/sec. Calculate the
throughput of your design.

\item Pipelined FWT processor (30\%). 
Modify your FWT processor so that it is pipelined and verify via
simulation. What is the new design's
maximum throughput? What is the speedup compared with the non-pipelined design?

\item Multicycle execution (40\%). 
It may not be feasible to supply the FWT processor with 64 high-speed,
parallel inputs. Develop a modified version of the FWT processor
which takes 2 inputs samples per cycle, i.e. it takes $N/2$ cycles
to obtain a complete input vector. Redesign your processor
so that it minimises the area-delay product (area being measured in
LUTs) and can process streaming input data without stalling.
What is the maximum performance in bytes/sec?

\item Comparison (bonus 20\%). 
Integrate the FWT processor with a waveform generator input source.
In real-time print out the FWT coefficients of the input signal.
What is the maximum speed that you can achieve?

\end{enumerate}

